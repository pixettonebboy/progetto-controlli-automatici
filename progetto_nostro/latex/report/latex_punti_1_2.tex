\documentclass{article}

\usepackage{graphicx} % per immagini
\usepackage{amsmath} % per matematica
\usepackage{amsfonts}   %per avere "font matematico"
\usepackage[a4paper, margin=2cm]{geometry} %margine di spazio dai bordi pagina (+ provarne altri)
\usepackage{booktabs}  % bordi belli tabella
\usepackage{tabularx}  % larghezza flessibile tabella
\usepackage{float}
\usepackage{mathtools} %per simboli matematici (come :=)
\usepackage[table]{xcolor} % colore righe
\usepackage{array}      % centrare verticalment
\usepackage[hidelinks]{hyperref}
\usepackage{tcolorbox}
\usepackage{siunitx} %per notazione scentifica
\usepackage{embedfile} %per inserire pdf (alla fine graphicx)

\title{
	\textbf{\LARGE Controlli Automatici T:}\\
	\textbf{\LARGE Controllo di un riscaldatore elettrico}\\[0.2cm]
	\large Progetto Tipologia b - Traccia 2
}
\author{
	Achille Pisani \\
	Alessandro Parmeggiani \\
	Youssef Esam Ebrahim Abou Aiesh
}

\date{}

%cambio il nome delle immagini da "Figure" a "Figura" e tabella=.
\renewcommand{\figurename}{Figura}
\renewcommand{\tablename}{Tabella}

\begin{document}
	
	%TAG UTILIZZATI: per figure "fig:", per equazioni "eq:", per sezioni "sec:", tabella "tab:"
	
	%pagina 1: titolo
	\maketitle
	\clearpage
	
	%pagina 2: sommario
	\tableofcontents
	\clearpage
	
	%metto sectionCounter-=1 per partire da cap. 0 e avere l'intro non coincidente con i punti.
	\setcounter{section}{-1}
	
	%sezione introduttiva ()
	\section{Introduzione} \label{sec:introduzione}
	
	\subsection{Descrizione del problema}
	
	\begin{figure}[htbp]
		\centering
		\includegraphics[width=0.4\textwidth]{figs/Schema-illustrativo-riscaldatore.png}
		\caption{Schema illustrativo del riscaldatore.}
		\label{fig:schemaScaldatore}
	\end{figure}
	
	Il sistema oggetto di studio è un riscaldatore elettrico che trasferisce calore a un flusso
d’aria in transito, come mostrato in Figura~\ref{fig:schemaScaldatore}.
La sua dinamica è descritta da due equazioni differenziali del primo ordine che descrivono
l’evoluzione della temperatura del riscaldatore e dell’aria in uscita:

	\begin{align}
		m_R c_R \frac{dT_R(t)}{dt} &= h_R A_R (T_{\text{out}}(t) - T_R(t)) + \frac{P_E(t)}{(1 + \kappa T_R(t))} \label{eq:diff1} \\
		m_A c_A \frac{dT_{\text{out}}(t)}{dt} &= \dot{m}_A c_A (T_{\text{in}} - T_{\text{out}}(t)) 
		+ h_R A_R (T_R(t) - T_{\text{out}}(t)). \label{eq:diff2}
	\end{align}
	
	\subsection{Parametri}
	
	I parametri forniti dalla traccia sono:
	\begin{itemize}
		\item $T_R(t)$ è la temperatura del riscaldatore [$^\circ$C];
		\item $T_{\text{out}}(t)$ è la temperatura dell'aria in uscita dal riscaldatore [$^\circ$C];
		\item $P_E(t)$ è la potenza elettrica fornita [$\mathrm{W}$];
		\item $T_{\text{in}}$ è la temperatura dell'aria in ingresso (ambiente a temperatura costante) [$^\circ$C];
		\item $m_R$ è la massa del riscaldatore [$\mathrm{kg}$];
		\item $c_R$ è il calore specifico del riscaldatore [$\mathrm{J/(kg \cdot ^\circ C)}$];
		\item $h_R$ è il coefficiente di convezione [$\mathrm{W/(m^2 \cdot ^\circ C)}$];
		\item $A_R$ è l’area di scambio termico tra riscaldatore e aria [$\mathrm{m^2}$];
		\item $\kappa$ è il coefficiente di variazione della resistenza con la temperatura [$1/^\circ\mathrm{C}$];
		\item $m_A$ è la massa dell'aria [$\mathrm{kg}$];
		\item $c_A$ è il calore specifico dell'aria [$\mathrm{J/(kg \cdot ^\circ C)}$];
		\item $\dot{m}_A$ è la portata massica dell’aria [$\mathrm{kg/s}$].
	\end{itemize}
	Si supponga di poter misurare la temperatura dell’aria in uscita dal riscaldatore $T_{\text{out}}(t)$ e di poter agire sulla potenza elettrica fornita al riscaldatore $P_E(t)$.
	\clearpage
	
	%tabella parametri
	\vspace*{\fill} %per metterla esattamente al centro della pagina
	\begin{table}[h!]    
		\centering
		\renewcommand{\arraystretch}{1.5} % altezza righe
		\label{tab:parametri}
		
		\rowcolors{2}{gray!15}{white} % righe alternate grigio chiaro e bianco
		\begin{tabularx}{0.9\textwidth}{>{\raggedright\arraybackslash}p{4cm} X}
			\toprule
			\rowcolor{gray!30} Parametro & Valore \\
			\midrule
			$h_R$              & $50~\mathrm{W/(m^2 \cdot ^\circ C)}$      \\
			$A_R$              & $0,07~\mathrm{m^2}$                        \\
			$c_R$              & $840,8~\mathrm{J/(kg \cdot ^\circ C)}$     \\
			$c_A$              & $1010~\mathrm{J/(kg \cdot ^\circ C)}$      \\
			$m_R$              & $2,542~\mathrm{kg}$                        \\
			$m_A$              & $0,1041~\mathrm{kg}$                        \\
			$\dot{m}_A$        & $0,2~\mathrm{kg/s}$                         \\
			$T_{\text{in}}$    & $28~^\circ\mathrm{C}$                        \\
			$\kappa$           & $3 \cdot 10^{-3} \, (1/^\circ\mathrm{C})$ \\
			$T_{\text{R,e}}$   & $175~^\circ\mathrm{C}$                       \\
			$T_{\text{out,e}}$ & $30,5036~^\circ\mathrm{C}$                        \\
			\bottomrule
		\end{tabularx}
		\caption{Valori dei parametri del riscaldatore}
	\end{table}
	\vspace*{\fill}
	\clearpage
	
	
	\section{Linearizzazione del sistema}
	\subsection{Richieste}
	
	
	\begin{tcolorbox}[
		colback=gray!10,
		colframe=black,
		left=8pt,
		right=8pt,
		top=8pt,
		bottom=8pt
		]
		
		Si riporti il sistema (\ref{fig:schemaScaldatore}) nella forma di stato
		\[
		\dot{x} = f(x, u) \tag{3a} \label{eq:3a}
		\]
		\[
		y = h(x, u). \tag{3b} \label{eq:3b}
		\]
		
		In particolare, si dettagli:
		\begin{itemize}
			\item la variabile di stato;
			\item la variabile d’ingresso;
			\item la variabile d’uscita;
			\item la forma delle funzioni \(f\) e \(h\).
		\end{itemize}
		
		A partire dai valori di equilibrio \(T_{R,e}\) e \(T_{\text{out},e}\) (forniti in tabella), si trovi l’intera coppia di equilibrio \((x_e, u_e)\) e si linearizzi il sistema non lineare (\ref{eq:sistema}) nell’equilibrio, così da ottenere un sistema linearizzato del tipo
		\[
		\delta\dot{x} = A\,\delta x + B\,\delta u \tag{4a} \label{eq:4a}
		\]
		\[
		\delta y = C\,\delta x + D\,\delta u \tag{4b} \label{eq:4b}
		\]
		con opportune matrici \(A\), \(B\), \(C\) e \(D\).
		
	\end{tcolorbox}
	
	%metodo alternativo per scrivere le variabili
	% \subsection{Variabili del sistema}
	
	% \begin{itemize}
		%     \item 
		%     \[
		%     x(t) = 
		%     \begin{bmatrix}
			%         x_1(t) \\
			%         x_2(t)
			%     \end{bmatrix}
		%     =
		%     \begin{bmatrix}
			%         T_R(t) \\
			%         T_{\text{out}}(t)
			%     \end{bmatrix}
		%     \]
		%     è il vettore di stato, che comprende:
		%     \begin{itemize}
			%         \item \(T_R(t)\): temperatura del riscaldatore \((^\circ\mathrm{C})\),
			%         \item \(T_{\text{out}}(t)\): temperatura dell’aria in uscita \((^\circ\mathrm{C})\).
			%     \end{itemize}
		
		%     \item 
		%     \[
		%     u(t) = P_E(t)
		%     \]
		%     è l’ingresso di controllo, ovvero la potenza elettrica fornita al riscaldatore \((\mathrm{W})\).
		
		%     \item 
		%     \[
		%     y(t) = T_{\text{out}}(t)
		%     \]
		%     è l’uscita del sistema, ovvero la temperatura dell’aria in uscita \((^\circ\mathrm{C})\).
		% \end{itemize}
	
	\subsection{Variabili del sistema}
	
	Le variabili individuate per descrivere il sistema in forma di stato sono:
	
	\begin{itemize}
		\item Lo stato:
		\[
		x(t) \coloneqq
		\begin{bmatrix}
			x_1(t) \\
			x_2(t)
		\end{bmatrix}
		=    
		\begin{bmatrix}
			T_R(t) \\
			T_{\text{out}}(t)
		\end{bmatrix}
		\]
		dove \(T_R(t)\) è la temperatura del riscaldatore \((^\circ\mathrm{C})\) e
		\(T_{\text{out}}(t)\) è la temperatura dell’aria in uscita \((^\circ\mathrm{C})\).
		
		\item L’ingresso:
		\[
		u(t) \coloneqq P_E(t)
		\]
		ovvero la potenza elettrica fornita al riscaldatore \((\mathrm{W})\).
		
		\item L’uscita:
		\[
		y(t) \coloneqq T_{\text{out}}(t) = x_2(t)
		\]
		ovvero la temperatura dell’aria in uscita \((^\circ\mathrm{C})\).
	\end{itemize}
	
	\subsection{Espressione del sistema in forma di stato}
	
	Il sistema può essere descritto mediante le seguenti equazioni:
	
	\begin{equation}
		\dot{x}(t) = f(x(t),u(t)) =
		\begin{bmatrix}
			f_1(T_R(t), T_{\text{out}}(t), u(t)) \\
			f_2(T_R(t), T_{\text{out}}(t), u(t))
		\end{bmatrix},
		\qquad
		y(t) = h(x(t),u(t)) =x_2(t)= T_{\text{out}}(t). \label{eq:sistema}
	\end{equation}
	
	Dove, nel dettaglio, le componenti di stato sono governate dalle seguenti equazioni:
	
	\[
	f_1(T_R, T_{\text{out}}, u)
	= 
	\frac{ a_R \left( T_{\text{out}} - T_R \right) + \displaystyle\frac{u}{1 + \kappa T_R} }
	{ m_R c_R },
	\]
	
	\[
	f_2(T_R, T_{\text{out}}, u)
	=
	\frac{ \dot{m}_A c_A (T_{\text{in}} - T_{\text{out}})
		+ a_R (T_R - T_{\text{out}}) }
	{ m_A c_A }.
	\]
	
	Pertanto, la forma di stato completa risulta:
	
	\[
	\dot{x}(t) =
	\begin{bmatrix}
		\displaystyle 
		\frac{ a_R (T_{\text{out}} - T_R) + \frac{u}{1+\kappa T_R} }
		{ m_R c_R }
		\\[10pt]
		\displaystyle
		\frac{
			\dot{m}_A c_A (T_{\text{in}} - T_{\text{out}}) + a_R (T_R - T_{\text{out}})
		}
		{ m_A c_A }
	\end{bmatrix},
	\qquad
	y(t) = T_{\text{out}}(t).
	\]
	
	\subsection{Analisi del sistema}
	Tramite le equazioni possiamo notare che il sistema è:
	\begin{itemize}
		\item \textbf{Tempo invariante}, poiché i coefficienti
        $h_R$ (coefficiente di convezione termica),
       $A_R$ (area di scambio),
       $m_R$ e $m_A$ (masse del riscaldatore e dell’aria),
       $c_R$ e $c_A$ (capacità termiche specifiche del riscaldatore e dell’aria),
       $\dot{m}_A$ (portata massica dell’aria),
        $\kappa$ (coefficiente di non linearità),
        $T_{\text{in}}$ (temperatura di ingresso dell’aria)
   sono assunti costanti.

		\item \textbf{SISO}, poichè l'equazione di uscita (\ref{eq:3b}) ha dimensione della variabile d' uscita : $p=1$ , e  la
		dimensione della variabile di ingresso è  $m = 1$ , quindi abbiamo che ($p=m=1$).
		\item \textbf{Strettamente proprio}, poichè $y(t)=T_{out}(t)$, quindi l'uscita non dipende dall'ingresso, ma solo dallo stato del sistema .
	\end{itemize}
	
	\subsection{Coppia di equilibrio}
	Dato un sistema tempo invariante continuo di tipo $\dot{x}(t) = f(x(t),u(t))$, abbiamo che \((x_e, u_e)\) è una coppia di equilibrio se $f(x_e, u_e)=0$.
    L’individuazione di una coppia di equilibrio è necessaria ai fini della linearizzazione
    del modello non lineare e corrisponde, dal punto di vista fisico, a una condizione di
    bilancio termico in cui la potenza elettrica fornita al riscaldatore compensa le perdite.
    di calore dovute allo scambio con l’aria.

	\subsubsection{Punto e ingresso di equilibrio}
	La traccia fornisce direttamente i valori di equilibrio delle variabili di stato.
    Definiamo quindi lo stato di equilibrio come:
	
	\begin{equation}
		x_e = 
		\begin{bmatrix}
			T_{R,e} \\[2mm]
			T_{out,e}
		\end{bmatrix}
		=
		\begin{bmatrix}
			175 ^\circ\mathrm{C} \\[2mm]
			30,5036 ^\circ\mathrm{C}
		\end{bmatrix}
	\end{equation}

  L’ingresso di equilibrio \(u_e\) può essere determinato imponendo l’annullamento
della derivata della temperatura del riscaldatore. In particolare, ponendo
\(\dot{T}_R = 0\) nell’equazione di stato associata al riscaldatore,
definendo la costante $K_{T_R,e} \, = \, 1+ K \, T_{R,e}$, otteniamo l'ingresso di equilibrio:

\begin{equation}
0 = h_R A_R \left( T_{\text{out},e} - T_{R,e} \right)
+ \frac{u_e}{1 + \kappa T_{R,e}} .
\end{equation}
Da cui segue:
\begin{equation}
u_e = P_{E,e}= -\, a_R \, (T_{\text{out},e} - T_{R,e}) \, K_{T_R,e}
\end{equation}
\begin{equation}
u_e = 771{,}250 \,\mathrm{W}
\end{equation}

	\subsubsection{Verifica dell'equilibrio nel modello non lineare}
	
	Per verificare che \((x_e, u_e)\) è una coppia di equilibrio, valutiamo le equazioni nel punto $\left(T_{R,e},\, T_{\text{out},e},\, u_e\right)$:
	
	\[
	f_1(T_{R,e}, T_{\text{out},e}, u_e), \qquad
	f_2(T_{R,e}, T_{\text{out},e}, u_e).
	\]
	Possiamo notare che corrispondono alle $\dot{T}_R$ e $\dot{T}_{\text{out}}$ del punto di equilibrio.
	
	Dai calcoli otteniamo che:
	
	\[
	f_1(T_{R,e}, T_{\text{out},e}, u_e) = 0 \; \text{[°C/s]},
	\qquad
	f_2(T_{R,e}, T_{\text{out},e}, u_e) = 9,7 \cdot 10^{-5}\ \; \text{[°C/s]}.
	\]
	Benchè il primo valore sia nullo, il secondo non è esattamente uguale a 0, ma, essendo un valore molto vicino allo 0 possiamo considerare soddisfatta la condizione e supporre \((x_e, u_e)\) coppia di equilibrio.
	Per procedere in maniera corretta senza supposizioni, possiamo prendere una tolleranza di $10^{-4}$.
    Perciò :la verifica numerica del modello non lineare nel punto (xe, ue) mostra che le derivate di
    stato risultano nulle entro la tolleranza numerica adottata. In particolare, la piccola discrepanza residua           osservata nella derivata di Tout è attribuibile agli errori di approssimazione
    numerica e non compromette la validità della coppia di equilibrio individuata.
	
	
	\subsection{Linearizzazione}
	
	Al fine di analizzare il comportamento del sistema in prossimità del punto di equilibrio
\((x_e, u_e)\) e di procedere alla progettazione del regolatore, il modello non lineare
viene linearizzato mediante uno sviluppo di Taylor al primo ordine.
	Se il sistema linearizzato risulta asintoticamente stabile intorno a questo punto, allora lo sarà anche quello non lineare.
	Pertanto possiamo scrivere il sistema nella seguente foma:
	
	\[
	\dot{x}(t) = A \, x(t) + B \, u(t), \qquad
	y(t) = C \, x(t) + D \, u(t),
	\]
	dove \( A \in \mathbb{R}^{2\times 2} \), \( B \in \mathbb{R}^{2\times 1} \),
	\( C \in \mathbb{R}^{1\times 2} \), \( D \in \mathbb{R}^{1\times 1} \).

Introducendo le variazioni rispetto al punto di equilibrio
\[
\delta x(t) = x(t) - x_e, \qquad
\delta u(t) = u(t) - u_e, \qquad
\delta y(t) = y(t) - y_e,
\]
con \(y_e = h(x_e, u_e)\), il sistema linearizzato assume la forma:
\begin{align}
\delta \dot{x}(t) &= A\,\delta x(t) + B\,\delta u(t), \\
\delta y(t) &= C\,\delta x(t) + D\,\delta u(t).
\end{align}

Le matrici \(A\), \(B\), \(C\) e \(D\) rappresentano rispettivamente le Jacobiane del modello
non lineare rispetto alle variabili di stato e all’ingresso, valutate nel punto di equilibrio:
\[
A = \left.\frac{\partial f(x,u)}{\partial x}\right|_{(x_e,u_e)}, \qquad
B = \left.\frac{\partial f(x,u)}{\partial u}\right|_{(x_e,u_e)}.
\]
    \[
C = \left.\frac{\partial h(x,u)}{\partial x}\right|_{(x_e,u_e)}, \qquad
D = \left.\frac{\partial h(x,u)}{\partial u}\right|_{(x_e,u_e)}.
\]

    \subsubsection{Calcolo delle jacobiane}
	Per effettuare la linearizzazione è necessario ricavare le jacobiane delle funzioni di stato \( f(x,u) \) rispetto alle variabili di stato e all’ingresso, valutandole nel punto di equilibrio \((x_e, u_e)\).
    Il calcolo numerico delle Jacobiane tramite differenze finite è giustificato dalla regolarità delle funzioni di stato. Il passo Δ è scelto sufficientemente piccolo in modo da garantire una buona approssimazione delle derivate parziali. 
    Nel presente elaborato il calcolo delle Jacobiane non è stato effettuato in forma analitica,
bensì tramite un’approssimazione numerica delle derivate parziali mediante il metodo delle
differenze finite centrate. In particolare, fissato un passo \(\Delta > 0\) sufficientemente
piccolo, le derivate sono state approssimate come:

\begin{align}
\left.\frac{\partial f_i}{\partial x_j}\right|_{(x_e,u_e)} &\approx
\frac{f_i(x_e + \Delta e_j, u_e) - f_i(x_e - \Delta e_j, u_e)}{2\Delta},
\qquad i,j = 1,2, \\
\left.\frac{\partial f_i}{\partial u}\right|_{(x_e,u_e)} &\approx
\frac{f_i(x_e, u_e + \Delta) - f_i(x_e, u_e - \Delta)}{2\Delta},
\qquad i = 1,2,
\end{align}

	
	La matrice di stato è:
	\begin{equation}
		A_e = \left.\frac{\partial f(x,u)}{\partial x}\right|_{(x_e,u_e)}
		=
		\begin{bmatrix}
			\dfrac{\partial f_1}{\partial T_R} & \dfrac{\partial f_1}{\partial T_{\text{out}}} \\[6pt]
			\dfrac{\partial f_2}{\partial T_R} & \dfrac{\partial f_2}{\partial T_{\text{out}}}
		\end{bmatrix}_{(x_e,u_e)}
		=
		\begin{bmatrix}
			-\num{2.1e-3} & \num{1.6e-3} \\
			0.0333 & -1.9545
		\end{bmatrix}.
	\end{equation}
	
	Da $A_{11}$ capiamo che la dinamica di $T_R$ è lenta; da $A_{22}$ capiamo che la dinamica di $T_{out}$ è veloce.
	
	La matrice d’ingresso è:
	\begin{equation}
		B_e = \left.\frac{\partial f(x,u)}{\partial u}\right|_{(x_e,u_e)}
		=
		\begin{bmatrix}
			\dfrac{\partial f_1}{\partial u} \\[6pt]
			\dfrac{\partial f_2}{\partial u}
		\end{bmatrix}_{(x_e,u_e)}
		=
		\begin{bmatrix} 
			
			\num{3.068e-4}\\
			0
		\end{bmatrix}.
	\end{equation}
	
	Da $B_{2}$ capiamo che l'ingresso non entra direttamente nell'equazione; da $B_{1}$ capiamo che in particolare, la potenza elettrica influisce direttamente solo sulla dinamica della temperatura del riscaldatore, mentre la temperatura dell’aria è influenzata indirettamente tramite lo scambio termico con il riscaldatore 
	
	Poiché la sola uscita del sistema è la temperatura \( T_{\text{out}} \), (sistema SISO) si ottiene:
	\begin{equation}
		C_e = \left.\frac{\partial h(x,u)}{\partial x}\right|_{(x_e,u_e)}
		= \begin{bmatrix} 0 & 1 \end{bmatrix},
		\qquad
		D_e = \left.\frac{\partial h(x,u)}{\partial u}\right|_{(x_e,u_e)}
		= 0.
	\end{equation}
	\( A_e \) 
	Il sistema linearizzato risulta dunque:
	\begin{equation}
		\delta \dot{x}(t) = A_e \, \delta x(t) + B_e \, \delta u(t),
		\qquad
		\delta y(t) = C_e \, \delta x(t) + D_e \, \delta u(t).
	\end{equation}
	
	\subsubsection{Autovalori e stabilità}

La stabilità locale del sistema linearizzato può essere analizzata studiando gli autovalori della matrice A. In particolare, il sistema linearizzato è asintoticamente stabile se e solo se tutti gli autovalori di A hanno parte reale negativa. La linearizzazione è valida localmente attorno al punto di equilibrio, che rappresenta una condizione di bilancio termico. Nel caso in esame, gli autovalori della matrice \( A_e \)  si trovano risolvendo il polinomio caratteristico :
	\[
	\det(A_e - \lambda I) = 0.
	\]
	
	Gli autovalori calcolati numericamente sono:
	\[
	\lambda_1 = -\num{2.1e-3}, \qquad
	\lambda_2 = -1.9545.
	\]
	
	Poiché entrambi presentano parte reale negativa, il sistema linearizzato risulta asintoticamente stabile nell’intorno del punto di equilibrio considerato. Si osserva inoltre che i due autovalori sono separati di diversi ordini di grandezza. Il polo λ1 è associato a una dinamica lenta, mentre il polo λ2 descrive una dinamica più rapida 
	\[
	\Re(\lambda_1) < 0, \qquad \Re(\lambda_2) < 0.
	\]
	
	Di conseguenza,La stabilità del sistema linearizzato garantisce la stabilità locale del modello non lineare
	attorno al punto di equilibrio.
	
	\section{Funzione di trasferimento}
	
	\subsection{Richieste}
	\begin{tcolorbox}[
		colback=gray!10,
		colframe=black,
		left=8pt,
		right=8pt,
		top=8pt,
		bottom=8pt
		]
		Si calcoli la funzione di trasferimento da $\delta u$ a $\delta y$, 
		ovvero la funzione $G(s)$ tale che
		$
		\delta Y(s) = G(s)\, \delta U(s).
		$
	\end{tcolorbox}
	
	\subsection{Definizione della funzione di trasferimento $G(s)$}
	
	Dato il modello nello spazio di stato linearizzato del sistema
	\[
	\dot{x}(t) = A \, x(t) + B \, u(t), \qquad
	y(t) = C \, x(t) + D \, u(t),
	\]
	
	è possibile definire la funzione di trasferimento $G(s)$ come il rapporto tra la trasformata di Laplace dell’uscita e quella dell’ingresso, assumendo condizioni iniziali nulle $x(0)=0$.
	
	Applicando la trasformata di Laplace alle equazioni del sistema linearizzato si ottiene:
\begin{align}
X(s) &= (sI - A)^{-1} x(0) + (sI - A)^{-1} B\,U(s), \label{eq:laplace_state} \\
Y(s) &= C(sI - A)^{-1} x(0) +
\left[ C(sI - A)^{-1} B + D \right] U(s). \label{eq:laplace_output}
\end{align}

La funzione di trasferimento G(s) rappresenta la relazione ingresso–uscita nel dominio di Laplace, assumendo condizioni iniziali nulle. 
Poiché l’uscita può essere espressa come il prodotto di convoluzione tra l’ingresso \(u(t)\)
e la risposta impulsiva \(g(t)\),
\[
y(t) = u(t) \ast g(t),
\]
nel dominio di Laplace vale la relazione:
\[
Y(s) = U(s)\,G(s).
\]

Considerando esclusivamente la risposta forzata del sistema e assumendo condizioni
iniziali nulle, ovvero \(x(0) = 0\), si ottiene:
\begin{equation}
Y_f(s) = \left[ C(sI - A)^{-1} B + D \right] U(s).
\end{equation}

Da cui segue la definizione della funzione di trasferimento:
\begin{equation}
G(s) = \frac{Y_f(s)}{U(s)} = C(sI - A)^{-1} B + D.
\end{equation}

Nel caso di sistemi SISO, la funzione di trasferimento assume la forma di un rapporto
tra polinomi:
\begin{equation}
G(s) = \frac{N(s)}{D(s)}.
\end{equation}

	dove il denominatore \(D(s) = \det(sI - A)\) coincide con il polinomio caratteristico
della matrice \(A\).

	
	\subsection{Calcolo della funzione di trasferimento $G(s)$}
	
	Considerata la funzione di trasferimento:
	\begin{equation}
		G(s)=\frac{\Delta T_{out}(s)}{\Delta PE(s)}
	\end{equation}
    La funzione di trasferimento del sistema linearizzato è data da:
\begin{equation}
G(s) = C_e (sI - A_e)^{-1} B_e + D_e.
\label{eq:tf_linearizzata}
\end{equation}

dove \(A_e\), \(B_e\), \(C_e\) e \(D_e\) sono le matrici del modello linearizzato valutate
nel punto di equilibrio. Nel caso in esame, l’inverso della matrice \((sI - A_e)\) risulta:
\begin{equation}
(sI - A_e)^{-1} =
\frac{1}{s^2 + 1.9566\,s + 0.004051}
\begin{bmatrix}
s+1.9545 & 1.6\cdot 10^{-3}\\[0.2cm]
0.0333 & s+2.1\cdot 10^{-3}
\end{bmatrix}.
\label{eq:inv_matrix}
\end{equation}



Sostituendo le matrici
\[

C_e =
\begin{bmatrix}
0 & 1
\end{bmatrix},
\qquad
B_e =
\begin{bmatrix}
3.068\cdot 10^{-4} \\
0
\end{bmatrix},
\qquad
D_e = 0.
\]

Si osserva che l’ingresso agisce direttamente solo sulla dinamica di $T_R$ (seconda componente nulla in $B_e$),
mentre $T_{out}$ è influenzata indirettamente tramite lo scambio termico.
asferimento della pianta:
\begin{equation}
G(s) = \frac{0.0103}{s^2 + 1.9566s + 0.0041}.
\label{eq:tf_pianta}
\end{equation}

Ponendo \(s = j\omega\), la funzione di trasferimento in frequenza assume la forma:
\begin{equation}
G(j\omega) = \left. \frac{0.0103}{s^2 + 1.9566s + 0.0041} \right|_{s = j\omega}.
\label{eq:tf_freq}
\end{equation}

	
	\subsubsection{Guadagno statico}
	Definiamo il guadagno statico della pianta come:
	
	\begin{equation}
		G(0) = \lim_{s \to 0} G(s)
		= \lim_{s \to 0} \left[ C (sI - A)^{-1} B + D \right]
		= -\,C A^{-1} B + D
		= 2,518\cdot10^{-3} \quad\frac{^\circ\mathrm{C}}{W}
	\end{equation}
    Rappresenta la variazione a regime della temperatura dell’aria in uscita in risposta a una variazione unitaria della potenza elettrica fornita al riscaldatore. 
    Il valore positivo del guadagno statico indica che un aumento della potenza elettrica fornita comporta un aumento della temperatura dell’aria in uscita, come atteso dal punto di vista fisico. 
	
	\subsubsection{Poli e zeri}
	Data la funzione di trasferimento %mettere poi label all'equation
	\[
	G(s)=\dfrac{N(s)}{D(s)}
	\]
	Definiamo come \textbf{poli} di $G(s)$ le radici del denominatore $D(s)$ e \textbf{zeri} le radici del numeratore $N(s)$.
	\\ Avremo che i poli di $G(s)$ sono:
	\begin{equation}
		p_1 = -1,9545 \ \frac{1}{s} \ , 
		\qquad 
		p_2 = -0,0021 \ \frac{1}{s}.
	\end{equation}
	
Nel caso in esame, non essendo presenti cancellazioni polo–zero, il denominatore della funzione di trasferimento coincide con il polinomio caratteristico della matrice A .
	Tutti i poli presentano parte reale negativa; il sistema è quindi BIBO stabile.
	Per quanto riguarda gli zeri di $G(s)$, si può notare che $G(s)$ non ha nessuno zero finito.

    Le caratteristiche dinamiche della funzione di trasferimento, in particolare la presenza di una dinamica lenta e di una rapida, saranno utilizzate nel seguito per il progetto del regolatore 
	
	\begin{figure}[H]
		\centering
		\includegraphics[width=0.95\textwidth]{figs/Punto2-Poli_e_Zeri_di_G(s).pdf}
	\end{figure}
	
	\subsubsection{Funzione di trasferimento finale}
	
	La funzione di trasferimento della pianta può quindi essere espressa come:
	\[
	G(s) = \frac{1{,}03 \cdot 10^{-5}}{(s + 1{,}9545)(s + 0{,}0021)}.
	\]
	
	Mostriamo il corrispondente diagramma di Bode:
	
	\begin{figure}[H]
		\centering
		\includegraphics[width=0.95\textwidth]{figs/Punto2-Bode_di_G(s).pdf}
	\end{figure}
	
	
	\section{Progettazione del regolatore}
	
	
	\subsection{Richieste}
	
	\begin{tcolorbox}[
		colback=gray!10,
		colframe=black,
		left=8pt,
		right=8pt,
		top=8pt,
		bottom=8pt
		]
		
		Si progetti un regolatore (fisicamente realizzabile) considerando le seguenti specifiche:
		\\\\
		1) Errore a regime $|e_{\infty}| \le e^{\star} = 0.002$ in risposta a un gradino
		$w(t) = W \cdot \mathbf{1}(t)$ e $d(t) = D \cdot \mathbf{1}(t)$ con ampiezze
		$W \le 4$ e $D \le 3.5$.
		\\\\
		2) Per garantire una certa robustezza del sistema si deve avere un margine di fase
		$M_f \ge 50^\circ$.
		\\\\
		3) Il sistema può accettare una sovraelongazione percentuale al massimo del $11\%$:
		$S\% \le 11\%$.
		\\\\
		4) Il tempo di assestamento alla $\varepsilon\% = 5\%$ deve essere inferiore al valore
		fissato: $T_{a,\varepsilon} = 0.01\,\mathrm{s}$.
		\\\\
		5) Il disturbo sull’uscita $d(t)$, con una banda limitata nel range di pulsazioni
		$[0,\,0.4]$, deve essere abbattuto di almeno $50\,\mathrm{dB}$.
		\\\\
		6) Il rumore di misura $n(t)$, con una banda limitata nel range di pulsazioni
		$[8 \cdot 10^{4},\, 9 \cdot 10^{6}]$, deve essere abbattuto di almeno $60\,\mathrm{dB}$.
		
	\end{tcolorbox}
	
	\subsection{Architettura e funzioni di sensitività}
	
	Progettiamo un regolatore $R(s)$ per l'impianto $G(s)$ in modo che le specifiche vengano rispettate.
	
	\begin{figure}[H]
		\centering
		\includegraphics[width=0.6\textwidth]{figs/Schema-di-controllo.png}
		\caption{Schema di controllo.}
		\label{fig:schemaDiControllo}
	\end{figure}
	
	Possiamo definire $L(s)=R(s)\cdot G(s)$, questa viene chiamata funzione d'anello.
	
	Inoltre, possiamo definire le funzioni di sensitività:
	\begin{itemize}
		\item $S(s)=\frac{1}{1+L(s)}$, ovvero la sensitività che controlla l'errore ed i disturbi all'ingresso ed all'uscita dell'impianto.
		
		\item $T(s)=\frac{L(s)}{1+L(s)}$, ovvero la sensitività che controlla il rumore di misure ed il riferimento ad alte frequenze.
	\end{itemize}
	
	Nel seguente grafico possono essere analizzate le funzioni di sensitività del sistema a confronto con la $L(s)$:
	
	\begin{figure}[H]
		\centering
		\includegraphics[width=0.95\textwidth]{figs/Punto3-Bode_L_S_T.pdf}
	\end{figure}
	
	\subsection{Specifiche}
	Le specifiche necessarie affinchè il regolatore progettato sia valido sono classificabili come:
	\begin{itemize}
		\item {Vincoli}
		\begin{itemize}
			\item Errore a regime $|e_{\infty}| \le e^{\star} = 0.002$.
			\item Ampiezza gradino $W_{max}=4$.
			\item Disturbo massimo $D_{max}=3,5$.
			\item Margine di fase minimo $M_{f,min} = 50^\circ$.
		\end{itemize}
		\item{Dinamici}
		\begin{itemize}
			\item Tempo di assestamento (5\%)\  $T^\star = 0,01 s$.
			\item Overshoot massimo $Sovr_{max}=0,11$
		\end{itemize}
		\item Disturbo (bassa frequenza)
		\begin{itemize}
			\item Attenuazione del disturbo $A_d = 50 \ dB$ in $[0 \ ,\ \omega_{d,max}]$, con $\omega_{d,max} = 0,4$.
		\end{itemize}
		\item Rumore (alta frequenza)
		\begin{itemize}
			\item Attenuazione del rumore $A_n = 60\ dB$ in $[\omega_{n,min}\ ,\ \omega_{n,max}]$, con $\omega_{n,min} = 8\cdot10^4$ e $\omega_{n,max} = 9\cdot10^6$
		\end{itemize}
		\item Specifiche Derivate:
		\begin{itemize}
			\item Smorzamento $\xi^\star$ calcolato dalla sovraelongazione massima facendo: $S\%=e^{-\frac{\pi\xi}{\sqrt{1-\xi^2}}}$, otterremo che $\xi^\star=\frac{|\ln(Sovr_{max})|}{\sqrt{\pi^2+\ln^2(Sovr_{max})}}$ così da avere $\xi^\star$ a margine di fase $M_f\approx100\xi^\star$.
			\item Margine di fase desiderato$ M_f$, trovato con: $M_f=\max(100\xi^\star \ , \ M_{f,min})$, dove prendiamo il massimo per rispettare il vincolo sul margine di fase minimo.
			\item Pulsazione crossover $\omega_{c,min}$ la otteniamo da margine di fase e tempo di assestamento: $\omega_{c,min}=\frac{300}{M_f\cdot T^\star}$.
		\end{itemize}
	\end{itemize}
	
	Per quanto riguarda le specifiche derivate, abbiamo come risultati:
	\[
	\xi^\star\approx0,575 \ , \qquad M_f\approx57,5^\circ \ , \qquad \omega_{c,min}=521,8\frac{rad}{s}.
	\]
	
	\subsection*{Sintesi regolatore}
	Il regolatore è realizzato come combinazione tra la sua parte statica e quella dinamica $R(s)=R_s(s)\cdot R_d(s)$.
	
	\subsection{Regolatore statico $R_s(s)$}
	
	Il regolatore statico si occupa delle specifiche riguardanti l’errore a regime.
	Scegliamo un guadagno $R_s$ per soddisfare i vincoli a bassa frequenza:
	
	\subsubsection{Vincolo sull'errore a regime}
	L'errore dovuto a un ingresso a gradino con retroazione unitaria dipende da $S(0)$.
	Per avere errore $e^\star$ con $W \le 4$ e $D \le 3,5$. %AGGIUSTARE QUESTA FRASE
	
	Nel caso peggiore: 
	\begin{equation}
		|e_{\infty}| \le |S(0)|(W_{max}+D_{max})
	\end{equation}
	Imponendo:
	\[
	|e_{\infty}| \le e^{\star}
	\]
	E assumendo:
	\[
	L(0)>>1
	\]
	Notiamo che:
	\[
	|L(0)|\ge\frac{W_{max}+D_{max}}{e^\star}
	\]
	E definiamo:
	\[
	\mu_{s,err}=\frac{W_{max}+D_{max}}{e^\star}
	\]
	Siccome $L(0)=R_sG(0)$, otteniamo 
	\[
	R_{s,err}=\frac{\mu_{s,err}}{|G(0)|}
	\]
	
	\subsubsection{Vincolo sul disturbo in bassa frequenza}
	L'attenuazione richiesta $A_d = 50 \ dB$ si traduce in:
	\[
	\mu_{s,dist}=10^{\frac{A_d}{20}}
	\]
	Imponiamo un loop abbastanza grande a $\omega_{d,max}$:
	\[
	|L(J\omega_{d,max})|\ge10^{\frac{A_d}{20}}
	\]
	Quindi:
	\[
	R_{s,dist}=\frac{\mu_{s,dist}}{|G(j\omega_{d,max})|}
	\]
	Scegliamo il massimo trai due per soddisfare entrambi i vincoli: $R_s=\max(R_{s,err}\ ,\ R_{s,dist})$
	
	I risultati da noi ottenuti sono:
	\[
	R_{s,err}\approx1,46\cdot10^6 \ , \qquad R_{s,dist}\approx2,47\cdot10^7
	\]
	Quindi prendiamo: 
	\[
	R_s=2,47\cdot10^7
	\]
	
	\subsection{Regolatore dinamico $R_d(s)$}
	Il regolatore dinamico non ha come obiettivo principale quello di "modificare" il modulo, ma può:
	\begin{itemize}
		\item Modificare fase e pendenza (ci serve per garantire $M_f \ge 50^\circ$ e crossover  $> \omega_{c,min}$).
		\item Ridurre il guadagno alle alte frequenze (ci serve per l'attenuazione del rumore $\ge 60\ dB$ in $[\omega_{n,min}\ ,\ \omega_{n,max}]$, con $\omega_{n,min} = 8\cdot10^4$ e $\omega_{n,max} = 9\cdot10^6$).
		\item Aumentare il guadagno alle basse frequenze (ci serve per l'attenuazione dei disturbi fino a $\omega_{d,max} = 0,4$)
	\end{itemize}
	\subsubsection{Polo per attenuazione del rumore di misura}
	Abbiamo introdotto un polo reale $R_{d,polo}(s)$ ad alta frequenza al fine di garantire l’attenuazione del rumore di misura.
	\begin{equation}
		R_{d,polo}(s)=\frac{1}{1+\frac{s}{\omega_{p,rumore}}}
	\end{equation}
	con:
	\[
	\omega_{p,rumore}=k_p\omega_{n,min} \ , \qquad k_p=0,3
	\]
	Prendiamo $k_p = 0,3$ per mettere il polo sotto la banda del rumore, in modo da attenuare il rumore senza tagliare la dinamica di controllo.
	
	\subsubsection{Rete anticipatrice $R_{lead} (s)$}
	Il fulcro del regolatore dinamico, serve perchè altrimenti il loop sarebbe troppo in ritardo in fase.
	Fissiamo $\omega_c^\star = 1,1 \omega_{c,min}$ (leggermente sopra il minimo per compensare approssimazioni).
	
	Ora, valutiamo modulo e fase dell'anello senza rete anticipatrice a $\omega_c^\star$, ovvero:
	\[
	L_0(j\omega_c^\star)\approx G_e(j\omega_c^\star)
	\]
	\[
	G_e(s)=R_sG(s) \ , \qquad G_e(j\omega_c^\star)
	\]
	e ricaviamo modulo:
	\[
	|G_e(j\omega_c^\star)|
	\]
	e fase:
	\[
	\angle G_e(j\omega_c^\star)|
	\]
	da cui ricaviamo:
	\begin{equation}
		PM_0 = 180^\circ+\angle G_e(j\omega_c^\star)    
	\end{equation}
	
	Nella figura sottostante si può analizzare $G_e(s)$ a confronto con $L(s)$:
	
	\begin{figure}[H]
		\centering
		\includegraphics[width=0.8\textwidth]{figs/Punto3-Bode_L(s)_Ge(s).pdf}
		\caption{Schema di controllo.}
		\label{fig:schemaDiControllo}
	\end{figure}
	
	I risultati da noi ottenuti sono:
	\[
	\angle G_e\approx-179,8^\circ \ , \qquad PM_0\approx0,2^\circ
	\]
	$PM_0$ è quasi nullo, pertanto imponiamo $PM_{des}$, ovvero un margine di fase desiderato, definito come: \\ $PM_{des}=M_f+8^\circ$, dove abbiamo aggiunto +8 per rendere il modello più robusto contro ad incertezze dovute ad approssimazioni.
	
	Pertanto, la fase che un lead deve aggiungere è: $\phi^\star=PM_{des}-PM_0$;
	in parallelo, imponiamo che a $\omega_c^\star$ il modulo sia $0\ dB$:
	\[|L(j\omega_c^\star)|=1\]
	Dato che $L\approx R_dG_e$ a quella frequenza:
	\[
	|R_d(j\omega_c^\star)|=\frac{1}{|G_e(j\omega_c^\star)|}=M^\star
	\]
	
	La rete lead scelta è:
	\[
	R_{d,antic}(s)=\frac{1+\tau s}{1+\alpha\tau s}, \qquad \text{con} \ 0<\alpha<1
	\]
	Abbiamo trovato un polo a $\alpha\tau$ ed uno zero a $\tau$.
	
	A partire dalle formule standard dello scenario B, arriviamo ad avere le seguenti espressioni:
	\begin{equation}
		\tau=\frac{M^\star-\cos\phi^\star}{\omega_c^\star\sin\phi^\star}\ , \qquad
		\alpha\tau=\frac{\cos\phi^\star-\frac{1}{M^\star}}{\omega_c^\star\sin\phi^\star}
	\end{equation}
	
	I risultati da noi ottenuti sono:
	\[
	PM_0\approx0,2^\circ\ , \qquad PM_{des}\approx65,5^\circ\ , \qquad \phi^\star\approx65,3^\circ\]\[
	M^\star \approx 1305,7\ , \qquad \tau\approx2,503 s\ , \qquad \alpha\tau\approx8\cdot10^{-4}s
	\]
	
	\subsubsection{Regolatore dinamico complessivo}
	Il regolatore dinamico complessivo risulta quindi:
	\[
	R_d(s) = R_{d,antic}(s)\,R_{d,polo}(s)
	\]
	
	Il regolatore dinamico risulta quindi composto da una rete anticipatrice,
	responsabile dell’incremento del margine di fase e della modellazione della
	dinamica del sistema, e da un polo ad alta frequenza, introdotto per garantire
	l’attenuazione del rumore di misura. L’aumento del guadagno alle basse frequenze,
	necessario per il rigetto dei disturbi, è invece demandato al regolatore statico.
	\subsection{Margini di L(s) e attenuazione S, T}
	
	\begin{itemize}
		\item Margine di guadagno $G_M = 32,05dB$.
		\item Margine di fase $P_M = 64,12^\circ$ a $\omega=573,87\frac{rad}{s}$.
		\item Disturbo (bassa frequenza) $s_{max}=\max|s(j\omega)| \le 10^{-\frac{Ad}{20}} = -53,08\ dB$.
		\item Rumore (alta frequenza) $T_{max}=\max|T(j\omega)| \le 10^{-\frac{An}{20}} = -89,3\ dB$.
	\end{itemize}
	
	\subsection{Prestazioni nel tempo (gradino) }
	
	
	Al fine di verificare il rispetto delle specifiche temporali, è stata analizzata la risposta al gradino del sistema in anello chiuso.  
	La funzione di trasferimento dal riferimento all’uscita è data da:
	\[
	F(s) = \frac{L(s)}{1 + L(s)},
	\]
	dove \( L(s) = R(s)G(s) \) rappresenta la funzione d’anello aperto.
	
	È stato applicato un gradino di riferimento di ampiezza \( W_{\max} \) e la risposta del sistema è stata ottenuta mediante simulazione temporale.
	
	\subsubsection{Tempo di assestamento}
	
	Il tempo di assestamento al 5\% è definito come l’istante oltre il quale l’uscita del sistema rimane permanentemente all’interno di una banda pari al \(\pm 5\%\) del valore di regime:
	\[
	0.95\,y_{\infty} \le y(t) \le 1.05\,y_{\infty}.
	\]
	
	Attraverso la funzione \texttt{stepinfo} di MATLAB, impostando una soglia di assestamento pari al 5\%, è stato calcolato il tempo di assestamento:
	\[
	T_s = 3.25 \cdot 10^{-3} \, \text{s}.
	\]
	
	Il valore ottenuto risulta inferiore al limite imposto dalla specifica:
	\[
	T_s \le 1.0 \cdot 10^{-2} \, \text{s},
	\]
	pertanto la specifica sulla rapidità del sistema risulta soddisfatta.
	
	\subsubsection{Sovraelongazione massima}
	
	La sovraelongazione percentuale è definita come:
	\[
	S = \frac{y_{\max} - y_{\infty}}{y_{\infty}} \cdot 100,
	\]
	dove \( y_{\max} \) rappresenta il valore massimo raggiunto dall’uscita durante il transitorio e \( y_{\infty} \) il valore di regime.
	
	Dall’analisi della risposta al gradino si ottiene:
	\[
	S = 4.98\%.
	\]
	
	Tale valore risulta ampiamente inferiore al limite massimo consentito dalla specifica:
	\[
	S \le 11\%,
	\]
	confermando un comportamento ben smorzato del sistema.
	
	
	\subsection{Regolatore finale}
	
	Il regolatore finale risulta essere:
	\begin{equation}
		R(s)=R_d(s)\cdot R_s
	\end{equation}
	Dopo aver fatto i calcoli risulterà essere:
	\begin{equation}
		R(s) =
		2.47\cdot10^{7}\;
		\frac{1 + 2.503\,s}
		{\left(1 + 8\cdot10^{-4}\,s\right)\left(1 + \dfrac{s}{2.4\cdot10^4}\right)}.
	\end{equation}
	Quindi la nostra $L(s) $sarà:
	\begin{equation}
		L(s)=R(s)G_e(s)=
		2.47\cdot10^{7}\;
		\frac{1 + 2.503\,s}
		{\left(1 + 8\cdot10^{-4}\,s\right)\left(1 + \dfrac{s}{2.4\cdot10^4}\right)}
		\cdot\frac{2{,}54 \cdot 10^{2}}
		{(s + 1{,}9545)(s + 0{,}0021)}.
	\end{equation}
	\clearpage
	
	\begin{figure}[H]
		\centering
		\includegraphics[width=0.95\textwidth]{figs/Punto3-Bode_di_L(s).pdf}
	\end{figure}
	
	\section{Test sul sistema linearizzato}
	
	\subsection{Richieste}
	
	\begin{tcolorbox}[
		colback=gray!10,
		colframe=black,
		left=8pt,
		right=8pt,
		top=8pt,
		bottom=8pt
		]
		
		Testare il sistema di controllo sul sistema linearizzato con $w(t) = 4 · 1(t)$, d(t) = $1.5 \sum^{4}_{k=1} sin(0.08kt)$ e
		$n(t) = 3 \sum_{k=1}^4 sin(5\cdot10^4kt)$.
		
	\end{tcolorbox}
	
	\subsection{Obiettivi}
	Andremo a verificare tramite simulazione nel tempo di un sistema lineare ad anello chiuso che il nostro regolatore soddisfi effettivamente le specifiche su:
	\begin{itemize}
		\item Riferimenti
		\item Disturbo
		\item Rumore
	\end{itemize}
	
	\subsection{Ingressi}
	Rappresentazioni grafiche degli ingressi forniti dalla richiesta, prima sullo scenario breve, poi su quello lungo.
	
	\begin{figure}[H]
		\centering
		\includegraphics[width=0.95\textwidth]{figs/Punto4-Scenario_breve_(Ingressi).pdf}
	\end{figure}
	
	\begin{figure}[H]
		\centering
		\includegraphics[width=0.95\textwidth]{figs/Punto4-Scenario_lungo_(Ingressi).pdf}
	\end{figure}
	
	\subsection{Modello ad anello chiuso e funzioni di trasferimento}
	
	Il test è stato condotto sul modello linearizzato del sistema, considerando il comportamento in anello chiuso rispetto ai tre ingressi previsti: riferimento $w(t)$, disturbo $d(t)$ e rumore di misura $n(t)$.
	
	In particolare, definite $L(s)$ come funzione di trasferimento in anello aperto, $T(s)$ come funzione di sensibilità complementare e $S(s)$ come funzione di sensibilità, valgono le seguenti relazioni:
	\[
	\begin{aligned}
		T_{wy}(s) &= T(s) \\
		T_{dy}(s) &= S(s) \\
		T_{ny}(s) &= -T(s)
	\end{aligned}
	\]
	
	dove il segno negativo nel trasferimento dal rumore all’uscita è dovuto al fatto che il rumore agisce sul ramo di retroazione.
	
	Queste funzioni di trasferimento vengono utilizzate per valutare separatamente l’effetto dei tre ingressi sull’uscita del sistema.
	
	\subsection{Margini di stabilità}
	
	I margini di stabilità sono stati calcolati sulla funzione di trasferimento in anello aperto $L(s)$ mediante il comando \texttt{margin}.  
	I risultati ottenuti sono:
	
	\begin{itemize}
		\item Margine di guadagno: $GM = 32.05$ dB
		\item Margine di fase: $PM = 64.12^\circ$
		\item Frequenza di attraversamento: $\omega_c = 573.87$ rad/s
	\end{itemize}
	
	Da tali valori capiamo che:
	\begin{itemize}
		\item Il margine di fase è maggiore del valore richiesto, specifica soddisfatta.
		\item Il margine di guadagno elevato indica una buona tolleranza a variazioni parametriche.
		\item La frequenza di crossover è coerente con la specifica $\omega_c^\star = 1,1\omega_{c,min}$
	\end{itemize}
	
	\subsection{Prestazioni sul gradino di riferimento}
	
	Le prestazioni dinamiche del sistema sono state valutate applicando un gradino di riferimento $\omega(t) = 4 \cdot 1(t)$ in assenza di disturbi e rumore ($d(t)=0$, $n(t)=0$).
	
	Otteniamo:
	
	\begin{itemize}
		\item Tempo di assestamento al 5\%: $T_s = 3.25 \cdot 10^{-3}$ s
		\item Overshoot: $4.98\%$
		\item Errore a regime: $6.43 \cdot 10^{-5}$ °C
	\end{itemize}
	
	Da questi dati evinciamo che:
	\begin{itemize}
		\item Il tempo di assestamento è molto inferiore al limite.
		\item L'overshoot è al di sotto del valore (11\%) richiesto.
		\item L'errore a regime è due ordini più piccolo del minimo richiesto.
	\end{itemize}
	
	Avremo che:
	\begin{equation}
		e_\infty=\omega_{max}|1-T(0)|=\omega_{max}|S(0)|
	\end{equation}
	Ciò conferma che $L(0)$ è grande come, imposto nel progetto del guadagno statico di $R_s$.
	
	\subsection{Scenario lungo: reiezione del disturbo a bassa frequenza}
	
	Per valutare la capacità del sistema di rigettare disturbi a bassa frequenza, è stata eseguita una simulazione su un orizzonte temporale lungo ($60$ s), applicando:
	
	\[
	d(t) = 1.5 \sum_{k=1}^{4} \sin(0.08 k t)
	\]
	
	Il rumore di misura è stato posto nullo.
	
	L’uscita dovuta al solo disturbo è stata calcolata facendo:
	\[
	y_{d1}(t) = T_{dy}(s)\, d(t)
	\]
	
	Il valore massimo assoluto osservato è:
	\[
	\max |y_{d1}(t)| = 0.006 \text{ °C}
	\]
	
	Il disturbo applicato ha ampiezza $1,5$ ed è composto da più sinusoidi lente, l'uscita presenta un'oscillazione molto ridotta rispetto all'ampiezza del disturbo ed è coerente con quanto progettato, confermando l’efficacia della funzione di sensibilità $S(s)$ nel rigettare disturbi a bassa frequenza, infatti:
	\begin{equation}
		S(s)=\frac{1}{1+L(s)}
	\end{equation}
	Che in bassa frequenza ha modulo inferiore a $-50 \ dB$.
	
	\subsection{Scenario breve: attenuazione del rumore di misura ad alta frequenza}
	
	Per analizzare l’effetto del rumore di misura, è stata effettuata una simulazione su un intervallo temporale breve ($0.05$ s) con passo molto ridotto, al fine di catturare le componenti ad alta frequenza.
	
	Ricordiamo che il rumore è definito come:
	\[
	n(t) = 3 \sum_{k=1}^{4} \sin(5 \cdot 10^4 k t)
	\]
	
	L’uscita dovuta al rumore è:
	\[
	y_{n2}(t) = T_{ny}(s)\, n(t)
	\]
	
	Il massimo valore assoluto osservato è:
	\[
	\max |y_{n2}(t)| = 5.18 \cdot 10^{-2} \text{ °C}
	\]
	
	Da questo possiamo evincere che:
	\begin{itemize}
		\item Il rumore ha ampiezza elevata ed è concentrato in alta frequenza.
		\item L'uscita mostra un'ampiezza molto più piccola confermando una forte attebuazione.
	\end{itemize}
	
	\subsection{Uscita totale del sistema}
	
	Avendo a disposizione i contributi all'uscita dati da riferimento, disturbi e rumore, possiamo ottenere l'uscita totale con:
	\begin{equation}
		y(t) = y_w(t) + y_d(t) + y_n(t)    
	\end{equation}
	
	
	\begin{figure}[H]
		\centering
		\includegraphics[width=0.95\textwidth]{figs/Punto4-Uscite_totali.pdf}
	\end{figure}
	
	Nel complesso, i risultati ottenuti confermano che il regolatore progettato soddisfa tutte le specifiche richieste.
	
	\section{Test sul sistema non lineare}
	\subsection{Richieste}
	
	\begin{tcolorbox}[
		colback=gray!10,
		colframe=black,
		left=8pt,
		right=8pt,
		top=8pt,
		bottom=8pt
		]
		
		\begin{itemize}
			\item Testare il sistema di controllo sul modello non lineare (ed in presenza di $d(t) \text{\ ed\ } n(t)$).
			\item Supponendo un riferimento $\omega(t)=2\cdot1(t)$, esplorare il range di condizioni iniziali dello stato del sistema non lineare (nell'intorno del punto di equilibrio) tali per cui l'uscita del sistema in anello chiuso converga a $h(x_e,u_e)$.
			\item Esplorare il range di ampiezza di riferimenti a gradino tali per cui il controllore rimane efficace sul sistema non lineare.
		\end{itemize}
	\end{tcolorbox}
	\subsection{Simulazione}
	Per lo svolgimento di questo punto abbiamo utilizzato Simulink.
	\subsubsection{Sistema ad anello chiuso non lineare}
	Quella che segue è la rappresentazione del sistema modellata utilizzando Simulink.
	\begin{figure}[H]
		\centering
		\includegraphics[width=0.95\textwidth]{figs/Punto5-Modello_Simulink.pdf}
	\end{figure}
	
	\subsubsection{Descrizione dei blocchi}
	
	\begin{itemize}
		\item $\omega(t)$: rappresenta il gradino d'ingresso.
		\item $n(t)$: rappresenta il rumore di misura.
		\item $R$: rappresenta il regolatore.
		\item $dx$: blocco MatLab che contiene l'equazione di stato.
		\item $u\_e$: rappresenta l'ingresso all'equilibrio.
		\item $\frac{1}{s}$: blocco Integrator che permette di integrare l'equazione differenziale.
		\item $C*u$: un gain che moltiplica l'uscita del blocco Integrator per la matrice $C$.
		\item $d(t)$: rappresenta il disturbo d'uscita.
		\item $-C-$: sottrazione da applicare per ottenere la non lineare.
		\item $y(t)$ (ultimo blocco a destra): Blocco scope che mostra l'uscita non lineare.
	\end{itemize}
	
	\subsubsection{Simulzaione con disturbi applicati}
	
	\clearpage
	\section{Punti opzionali}
	
	
	
\end{document}
