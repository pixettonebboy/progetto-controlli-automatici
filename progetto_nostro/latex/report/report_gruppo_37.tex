\documentclass{article}

\usepackage{graphicx} % per immagini
\usepackage{amsmath} % per matematica
\usepackage[a4paper, margin=2cm]{geometry} %margine di spazio dai bordi pagina (+ provarne altri)
\usepackage{booktabs}  % bordi belli tabella
\usepackage{tabularx}  % larghezza flessibile tabella

\usepackage[table]{xcolor} % colore righe
\usepackage{array}      % centrare verticalment

\title{
    \textbf{\LARGE Relazione Controlli Automatici T:}\\
    \textbf{\LARGE Controllo di un riscaldatore elettrico}\\[0.2cm]
    \large Progetto Tipologia b - Traccia 2
    }
    \author{
        Achille Pisani \\
        Alessandro Parmeggiani \\
        Youssef Esam Ebrahim Abou Aiesh
    }

    \date{}

%cambio il nome delle immagini da "Figure" a "Figura" e tabella=.
\renewcommand{\figurename}{Figura}
\renewcommand{\tablename}{Tabella}

\begin{document}

%TAG UTILIZZATI: per figure "fig:", per equazioni "eq:", per sezioni "sec:", tabella "tab:"

%pagina 1: titolo
\maketitle
\clearpage

%pagina 2: sommario
\tableofcontents
\clearpage

%metto sectionCounter-=1 per partire da cap. 0 e avere l'intro non coincidente con i punti.
\setcounter{section}{-1}

%sezione introduttiva ()
\section{Introduzione} \label{sec:introduzione}

\subsection{Descrizione del problema}

\begin{figure}[htbp]
    \centering
    \includegraphics[width=0.4\textwidth]{figs/Schema-illustrativo-riscaldatore.png}
    \caption{Schema illustrativo del riscaldatore.}
    \label{fig:schemaScaldatore}
\end{figure}

Si consideri il sistema in Figura~\ref{fig:schemaScaldatore} rappresentante un riscaldatore elettrico che riscalda dell’aria in transito.
La dinamica del sistema è descritta dalle seguenti equazioni differenziali:
\begin{align}
m_R c_R \frac{dT_R(t)}{dt} &= h_R A_R (T_{\text{out}}(t) - T_R(t)) + \frac{P_E(t)}{(1 + \kappa T_R(t))} \label{eq:diff1} \\
m_A c_A \frac{dT_{\text{out}}(t)}{dt} &= \dot{m}_A c_A (T_{\text{in}} - T_{\text{out}}(t)) 
+ h_R A_R (T_R(t) - T_{\text{out}}(t)). \label{eq:diff2}
\end{align}

\subsection{Parametri}

I parametri forniti dalla traccia sono:
\begin{itemize}
    \item $T_R(t)$ è la temperatura del riscaldatore [$^\circ$C];
    \item $T_{\text{out}}(t)$ è la temperatura dell'aria in uscita dal riscaldatore [$^\circ$C];
    \item $P_E(t)$ è la potenza elettrica fornita [$\mathrm{W}$];
    \item $T_{\text{in}}$ è la temperatura dell'aria in ingresso (ambiente a temperatura costante) [$^\circ$C];
    \item $m_R$ è la massa del riscaldatore [$\mathrm{kg}$];
    \item $c_R$ è il calore specifico del riscaldatore [$\mathrm{J/(kg \cdot ^\circ C)}$];
    \item $h_R$ è il coefficiente di convezione [$\mathrm{W/(m^2 \cdot ^\circ C)}$];
    \item $A_R$ è l’area di scambio termico tra riscaldatore e aria [$\mathrm{m^2}$];
    \item $\kappa$ è il coefficiente di variazione della resistenza con la temperatura [$1/^\circ\mathrm{C}$];
    \item $m_A$ è la massa dell'aria [$\mathrm{kg}$];
    \item $c_A$ è il calore specifico dell'aria [$\mathrm{J/(kg \cdot ^\circ C)}$];
    \item $\dot{m}_A$ è la portata massica dell’aria [$\mathrm{kg/s}$].
\end{itemize}
Si supponga di poter misurare la temperatura dell’aria in uscita dal riscaldatore $T_{\text{out}}(t)$ e di poter agire sulla potenza elettrica fornita al riscaldatore $P_E(t)$.
\clearpage
%tabella parametri
\begin{table}[h!]
    \centering
    \renewcommand{\arraystretch}{1.5} % altezza righe
    \label{tab:parametri}
    \rowcolors{2}{gray!15}{white} % righe alternate grigio chiaro e bianco
    \begin{tabularx}{0.9\textwidth}{>{\raggedright\arraybackslash}p{4cm} X}
        \toprule
        \rowcolor{gray!30} Parametro & Valore \\
        \midrule
        $h_R$              & $50~\mathrm{W/(m^2 \cdot ^\circ C)}$      \\
        $A_R$              & $0,07~\mathrm{m^2}$                        \\
        $c_R$              & $840,8~\mathrm{J/(kg \cdot ^\circ C)}$     \\
        $c_A$              & $1010~\mathrm{J/(kg \cdot ^\circ C)}$      \\
        $m_R$              & $2,542~\mathrm{kg}$                        \\
        $m_A$              & $0,1041~\mathrm{kg}$                        \\
        $\dot{m}_A$        & $0,2~\mathrm{kg/s}$                         \\
        $T_{\text{in}}$    & $28~^\circ\mathrm{C}$                        \\
        $\kappa$           & $3 \cdot 10^{-3} \, (1/^\circ\mathrm{C})$ \\
        $T_{\text{R,e}}$   & $175~^\circ\mathrm{C}$                       \\
        $T_{\text{out,e}}$ & $80~^\circ\mathrm{C}$                        \\
        \bottomrule
    \end{tabularx}
    \caption{Valori dei parametri del riscaldatore}
\end{table}
\clearpage



\end{document}