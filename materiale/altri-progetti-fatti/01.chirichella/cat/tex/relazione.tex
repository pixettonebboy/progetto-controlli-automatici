\documentclass[a4paper]{article}
\usepackage[utf8]{inputenc}
\usepackage[left=2cm, right=2cm]{geometry}
\usepackage{amsmath}
\usepackage{amssymb}
\usepackage{graphicx}
\usepackage{subcaption}
\usepackage{float}
\usepackage{gensymb}
\usepackage{hyperref}
\usepackage{hyperref}
\hypersetup{
    colorlinks=true,
    linkcolor=black,
    citecolor=green,
    filecolor=magenta,
    urlcolor=cyan,
    pdfborder={0 0 0}
}

\renewcommand{\contentsname}{Contenuti}

\title{
    \textbf{\LARGE Relazione Controlli Automatici T:}\\
    \textbf{\LARGE Controllo di una Tavola Rotante Motorizzata}\\[0.2cm]
    \large Progetto Tipologia b - Traccia 3
    }
    \author{Davide Chirichella, Filippo Giulietti, Renato Eugenio Maria Marziano, Michele Proietti}
    \date{}
\numberwithin{equation}{section}
\begin{document}

\maketitle
\tableofcontents
\clearpage

\section{Introduzione}
\subsection{Descrizione del problema}
Il sistema è descritto dall'equazione differenziale:
\begin{equation}
J\dot{\omega} = \tau(\theta)C_m - \beta\omega - k\theta,
\end{equation}
con rapporto di trasmissione $\tau(\theta)$ dato da:
\begin{equation}
\tau(\theta) = \frac{\cos(\alpha)}{1 - (\sin(\alpha)\cos(\theta))^2}.
\end{equation}

\subsection{Parametri}
I parametri forniti dalla traccia sono:
\begin{itemize}
    \item $k$: elasticità del disco ($k = 500 \ \mathrm{N/m^2}$),
    \item $J$: momento d'inerzia della tavola ($J = 400 \ \mathrm{J \cdot m^2}$),
    \item $\beta$: coefficiente di attrito viscoso ($\beta = 0.5 \ \mathrm{N \cdot s/m^2}$),
    \item $\alpha$: angolo del giunto cardanico ($\alpha = 50^\circ$).
    \item $\alpha$: posizione angolare di equilibrio ($\theta_e = 100^\circ$).
\end{itemize}

\section{Analisi del sistema}

\subsection{Variabili del sistema}
Le variabili individuate per descrivere il sistema in forma di stato sono:
\begin{itemize}
    \item $x(t) = \begin{bmatrix}x_1(t) \\ x_2(t)\end{bmatrix} = \begin{bmatrix}\theta(t) \\ \omega(t)\end{bmatrix} = \begin{bmatrix}\theta(t) \\ \dot{\theta(t)}\end{bmatrix} $ è lo stato che comprende la posizione angolare e la velocità angolare della tavola,
    \item $u(t) = C_m$ è l'ingresso di controllo, ovvero la coppia generata dal motore elettrico,
    \item $y(t) = \theta(t)$ è l'uscita del sistema, ovvero la posizione angolare della tavola.
\end{itemize}

\subsection{Forma di stato}
Il sistema è stato riscritto in forma di stato con le seguenti equazioni:
\begin{align}
\dot{x}(t) &= f(x(t),u(t)) = \begin{bmatrix} 
x_2(t) \\ 
\frac{\tau(\theta)C_m - \beta x_2(t) - k x_1(t)}{J} 
\end{bmatrix} \\
y(t) &= h(x(t)) = x_1(t)
\end{align}

\subsection{Considerazioni sul sistema}
Dalle equazioni emerge che si tratta di un sistema:
\begin{itemize}
    \item \textbf{Non lineare}, in quanto la sua dinamica contiene elementi non lineari (seno, coseno). Necessiterà quindi di linearizzazione,
    \item \textbf{Forzato}, in quanto la u(t) è presente nelle equazioni di stato,
    \item \textbf{Tempo invariante}, in quanto i parametri in gioco sono costanti,
    \item \textbf{Single Input Single Output (SISO)}, in quanto sia l'ingresso che l'uscita hanno dimensione 1,
    \item \textbf{Strettamente proprio}, in quanto l'uscita dipende solo dallo stato e non dall'ingresso.
\end{itemize}

\subsection{Linearizzazione}
Essendo un sistema forzato, l'equilibrio è rappresentato come una coppia \((u_e, x_e)\). Sapendo che \(\theta_e = 100^\circ\), l'equilibrio del sistema è stato calcolato come:
\begin{align}
x_e &= \begin{bmatrix} 
\theta_e \\ 
0 
\end{bmatrix} = 
\begin{bmatrix} 
100^\circ \\ 
0 
\end{bmatrix}, \\
u_e &= C_{m,e} = \frac{k \theta_e}{\tau(\theta_e)} \approx 1333.6021.
\end{align}
L'operazione di linearizzazione ci permetterà di ottenere un sistema lineare che approssima il comportamento dinamico del sistema originario (non lineare) 
attorno all’equilibrio, ottenendo la forma di stato linearizzata nella struttura:
\begin{align}
    \Delta \dot{x}(t) &= A_e \Delta x(t) + B_e \Delta u(t) \\
    \Delta y(t) &= C_e \Delta x(t) + D_e \Delta u(t)
\end{align}
Le matrici \( A_e, B_e, C_e, D_e \) sono linearizzate attorno alla coppia di equilibrio. Esse si ottengono calcolando le derivate parziali delle funzioni di stato e di uscita valutate nell'equilibrio rispetto alle variabili di stato e di ingresso, 
come segue:

\begin{align*}
A_e &= \frac{\partial f(x,u)}{\partial x} \bigg|_{x = x_e, u = u_e} = \begin{bmatrix}
0 & 1 \\
-\frac{k}{J} & -\frac{\beta}{J}
\end{bmatrix} \\
B_e &= \frac{\partial f(x,u)}{\partial u} \bigg|_{x = x_e, u = u_e} = \begin{bmatrix}
0 \\
\frac{\tau(\theta_e)}{J}
\end{bmatrix} \\
C &= \begin{bmatrix}
1 & 0
\end{bmatrix} \\
D &= 0
\end{align*}
Le matrici linearizzate sono quindi:
\begin{align*}
A_e &= \frac{\partial f(x,u)}{\partial x} \bigg|_{x = x_e, u = u_e} = \begin{bmatrix}
0 & 1 \\
-0.8042 & -0.0013
\end{bmatrix} \\
B_e &= \frac{\partial f(x,u)}{\partial u} \bigg|_{x = x_e, u = u_e} = \begin{bmatrix}
0 \\
0.0016
\end{bmatrix} \\
\end{align*}

\subsection{Funzione di trasferimento}
La funzione di trasferimento del sistema \(G(s) = C(sI - A_e)^{-1}B_e + D\) è stata calcolata come:
\begin{equation}
    G(s) \approx \frac{2.0343 \cdot  10^{-3}}{ 1 + 1.5543 \cdot 10^{3} s + 0.8042 \cdot 10^{-1} s^2 } % Parte corretta
\end{equation}

\begin{figure}[H]
\centering
\includegraphics[width=1\textwidth]{figs/Bode_G(s).jpeg}
\caption{Diagramma di Bode della funzione di trasferimento $G(j\omega)$}
\label{fig:etichetta}
\end{figure}
La funzione di trasferimento risultante presenta \textbf{due poli complessi coniugati}, come evidenziato sia dalla forma caratteristica del diagramma di Bode che 
dalla posizione dei poli nel piano complesso. Presenta inoltre del \textbf{guadagno statico} di tipo amplificativo. Inoltre, poiché i poli hanno una parte reale strettamente negativa, il sistema soddisfa i requisiti per la \textbf{stabilità esterna} (BIBO stabilità) garantendoci in uscita una funzione limitata dando in ingresso un segnale a sua volta limitato. 

\section{Specifiche per la realizzazione del regolatore}
Nei punti precedenti abbiamo descritto il comportamento del nostro sistema ottenendo la funzione di trasferimento $G(s)$. Il prossimo passo è quello di progettare un regolatore con funzione di trasferimento $R(s)$ che ci dia garanzie di stabilità, precisione, robustezza etc... rispettando le specifiche date nella traccia.
$L(s) = G(s)R(s)$ non è altro che il regolatore messo in serie con il sistema, definita come \textbf{funzione ad anello aperto}.
\subsection{Specifiche di progetto}
Le specifiche del regolatore da progettare sono:
\begin{enumerate}
    \item Errore a regime $\lvert e_\infty \rvert \leq e^\star = 0.01$ in risposta a un gradino $w(t) = 1.5(t)$ e $d(t) = 1(t)$;
    \item Per garantire una certa robustezza del sistema si deve avere un margine di fase $M_f \geq 33^\circ$;
    \item Il sistema può accettare una sovraelongazione percentuale al massimo dell'16\%: $S\% \leq 16\%$;
    \item Il tempo di assestamento alla $\epsilon \% = 5\%$ deve essere inferiore al valore fissato: $T_{a,\epsilon} \leq 0.003 \ \mathrm{s}$;
    \item Il disturbo sull'uscita $d(t)$, con una banda limitata nel range di pulsazioni $[0, 0.8]$, deve essere abbattuto di almeno $50 \ \mathrm{dB}$;
    \item Il rumore di misura $n(t)$, con una banda limitata nel range di pulsazioni $[1.2 \cdot 10^5, 5 \cdot 10^6]$, deve essere abbattuto di almeno $72 \ \mathrm{dB}$.
\end{enumerate}

\subsection{Regolatore statico}
\subsubsection{Errore a regime}
Aggiungere uno zero avrebbe annullato completamente l'errore a regime, noi invece vogliamo semplicemente
mantenerlo sotto una certa soglia. Per ottenere un errore a regime $|e_\infty| \leq e^* = 0.01$ in risposta a un gradino $w(t) = 1.5$ e $d(t) = 1(t)$, agiamo sul
guadagno della funzione di trasferimento:

\begin{equation}
    \mu_\text{min} = L(0) \geq \frac{D^\star + W^\star}{e^\star} = 250
\end{equation}

\subsection{Regolatore dinamico}
\subsubsection{Sovraelongazione percentuale e Margine di fase}
La sovraelongazione percentuale indica quanto l'uscita del sistema supererà il valore desiderato a regime prima di stabilizzarsi.
Per ottenere una sovraelongazione percentuale massima $S\% \leq S^\star = 16\% \ $, con un coefficiente di smorzamento $\xi$ calcolato come 
$\xi \approx \frac{M_f}{100}$, bisognerà avere un margine di fase desiderato \(M_f^\star\) circa uguale a:
\begin{equation}
    M_f^\star \geq 100 \cdot \xi^\star = \frac{\left| \log\left( \frac{S^\star}{100} \right) \right|}{\sqrt{\pi^2 + \left( \log\left( \frac{S^\star}{100} \right) \right)^2}} \approx 50.3868^\circ
\end{equation}

\subsubsection{Tempo di assestamento e frequenza critica minima \( \omega_{c,\text{min}} \)}
Il tempo di assestamento per un errore relativo \( \varepsilon\% = 5\% \) deve essere inferiore al valore fissato $T_{a, \epsilon} \leq T^\star = 0.003 \, \text{s}$.
Per garantire questa condizione, deve essere soddisfatta la relazione:
\begin{equation}
    \xi \cdot \omega_n \geq \frac{3}{T^\star} \implies M_f \cdot \omega_c \geq \frac{3 \cdot 100}{T^\star}
\end{equation}
Con $\omega_n$ pulsazione naturale e $\xi$ coefficiente di smorzamento. Possiamo quindi ricavare la frequenza minima come:
\[
\omega_{c, \text{min}} = \frac{300}{M_f^\star \cdot T^\star} \approx  1984.6464
\]
Dove \( M_f ^ {\star} \) è il margine di fase scelto come il più restrittivo tra quelli imposti nel punto precedente dalla disequazione nel vincolo di sovraelongazione.

\subsubsection{Disturbo sull'uscita} 
Sappiamo che se $d(t)$ è una sinusoide di equazione $ d(t) = D cos ( \omega t + \varphi) $, allora $y(t) = |S(j\omega)|D \cos\left(\omega t + \varphi - \arg\{S(j\omega)\}\right)$
, con $S(j\omega)$ definita \textbf{funzione di sensitività}. Grazie all'analisi in frequenza della funzione di sensitività possiamo approssimare la $S(j \omega)$ come segue: 
\begin{equation}
|S(j\omega)|_{\text{dB}} \approx 
\begin{cases} 
-|L(j\omega)|_{\text{dB}} & \omega \leq \omega_c \\
0 \ \text{dB} & \omega > \omega_c
\end{cases}
\end{equation}
Le specifiche richiedono di attenuare il disturbo sull'uscita, con una banda limitata nel range di pulsazioni $ [ \omega_{d, \min} = 0, \omega_{d, \max} = 0.8] $, di almeno $50\mathrm{dB}$. Poichè la $\omega_{d, \max} \ll \omega_c $ calcolata precedentemente, ricadiamo nel primo caso e possiamo affermare che:
\begin{equation}
|S(j\omega)|_{\text{dB}} \leq {-A_d} \implies |L(j\omega)|_{\text{dB}} \geq {A_d} \implies |L(j\omega)|_{\text{dB}} \geq 50 dB
\end{equation}
$L(s)$ deve avere quindi un guadagno di almeno $50\mathrm{dB}$ all'intervallo di frequenze del disturbo

\subsubsection{Disturbo di misura}
Per le stesse ragioni precedenti, sappiamo che se $n(t)$ è una sinusoide di equazione $ n(t) = N cos( \omega t + \varphi) $, allora $y(t) = |F(j\omega)|N \cos\left(\omega t + \varphi - \arg\{F(j\omega)\}\right)$, con $F(j\omega)$ \textbf{funzione di sensitività complementare}.
Il range di pulsazioni considerato nel caso del disturbo di misura è $ [ \omega_{n, \min} = 1.2 \cdot 10^5 , \omega_{n, \max} = 5 \cdot 10^6 ] $. Possiamo approssimare la $F(j \omega)$ come segue:
\begin{equation}
|F(j\omega)|_{\text{dB}} \approx 
\begin{cases} 
0 \ \text{dB} & \omega < \omega_c \\
|L(j\omega)|_{\text{dB}} & \omega \geq \omega_c
\end{cases}
\end{equation}
Il disturbo di misura va attenuato di almeno $72\mathrm{dB}$. Poichè la $\omega_{d, \max} \gg \omega_c $, ricadiamo nel secondo caso e possiamo affermare che:
\begin{equation}
|F(j\omega)|_{\text{dB}} \leq {-A_n} \implies |L(j\omega)|_{\text{dB}} \leq {-A_n} \implies |L(j\omega)|_{\text{dB}} \leq -72 dB
\end{equation}
$L(s)$ deve avere quindi un guadagno inferiore ad almeno $72\mathrm{dB}$ all'intervallo di frequenze del disturbo

\subsubsection{G(s) con aree proibite evidenziate}
Mostriamo ora graficamente quali aree il nostro regolatore dovrà impedire di oltrepassare:
\begin{figure}[H]
\centering
\includegraphics[width=1\textwidth]{figs/Bode_G(s)_patch.jpeg}
\caption{Diagramma di Bode della funzione di trasferimento $G(j\omega)$ con le aree proibite }
\label{fig:etichetta}
\end{figure}
\begin{itemize}
    \item La patch \textbf{rosa} identifica la zona proibita relativa al vincolo sulla riduzione del disturbo di uscita $d(t)$;
    \item La patch \textbf{verde} nel diagramma delle ampiezze identifica la zona proibita relativa al vincolo sulla riduzione degli errori di misura $n(t)$;
    \item La patch \textbf{blu} identifica la zona proibita relativa alla pulsazione di taglio minima $ \omega_\text{c,min}$, ovvero le pulsazioni in cui la funzione non può andare in $0\mathrm{dB}$;
    \item La patch \textbf{verde} nel diagramma delle fasi identifica la zona proibita relativa al margine di fase $ M_f^{\star} $, ovvero la zona che non può essere attraversata nella pulsazione di taglio ;
\end{itemize}
\section{Sintesi regolatore}
Il regolatore è realizzato come combinazione tra la sua parte statica e quella dinamica $ R(s) = R_s(s) \cdot R_d(s) $
\subsection{Sintesi del regolatore statico}

%% REGOLATORE STATICO
\section*{Regolatore Statico}
\subsubsection{Sintesi per errore a regime}
Il guadagno statico deve essere maggiore o uguale al minimo calcolato nel punto 3.2.1 per rispettare l'errore a regime:
\begin{equation}
\mu_{s,\text{error}} \geq \mu_\text{min} \approx 250
\end{equation}
\subsubsection{Sintesi per abbattimento disturbi}
Applichiamo la formula per il guadagno statico utilizzato per l'abbattimento dei disturbi ($A_d$ = attenuazione dei disturbi, 50):
\begin{equation}
\mu_{s,\text{dist}} = 10^{A_d/20} \approx 316.2278
\end{equation}
\subsubsection{Sintesi regolatore statico completa}
Otteniamo che la parte statica del regolatore ha una funzione di trasferimento del tipo:
\begin{equation}
R_s(s) = \max\left(\frac{\mu_{s,\text{error}}}{G(0)}, \frac{\mu_{s,\text{dist}}}{G( j \omega_\text{d,max})}\right) \approx 1.2290 \cdot 10^5
\end{equation}

\subsubsection{Rappresentazione funzione estesa}
Rappresentiamo ora il diagramma di Bode della funzione estesa, ovvero la serie tra il sistema e la parte statica del regolatore:
\begin{figure}[H]
\centering
\includegraphics[width=1\textwidth]{figs/Bode_G(s)_estesa.jpeg}
\caption{Diagramma di Bode della funzione di trasferimento estesa $G_e(j\omega) = G(j\omega)R_s(j\omega)$ }
\label{fig:etichetta}
\end{figure}

%% REGOLATORE DINAMICO
\subsection{Sintesi del regolatore dinamico}
\subsubsection{Sintesi per il rispetto del margine di fase}
Osservando il diagramma di Bode del sistema esteso notiamo che ci troviamo nello scenario \textbf{B} studiato a lezione, in quanto non esistono pulsazioni in cui la fase del regolatore esteso $ G_e(j\omega) $ sodisfi il vincolo sul margine di fase. Il nostro obiettivo è quindi di aumentare la fase nell'intervallo critico in modo da rispettare il vincolo su $ M_f^{\star}$,
amplificando il meno possibile l'ampiezza. Le strategie da adottare sono:
\begin{itemize}
    \item Aggiungere uno o più \textbf{zeri} per aumentare la fase;
    \item Aggiungere uno o più \textbf{poli} per mitigare l'aumento dell'ampiezza;
\end{itemize}
Tenendo conto dell'aggiunta di uno o più poli, progettiamo una rete anticipatrice con la seguente funzione di trasferimento:
\begin{equation}
R_\text{d,anticipatrice} = \frac{ 1 + \tau s}{1 + \alpha \tau s}
\end{equation}
dove $ \tau $ rappresenta il tempo caratteristico che definisce la posizione del polo e dello zero, e $ \alpha $, compreso tra 0 e 1, regola il rapporto tra le loro rispettive posizioni.
Ricordando le formule di inversione:
\begin{align*}
\tau &= \frac{\cos\varphi^\star - \frac{1}{M^\star}}{\omega^\star_c \sin\varphi^\star} \\
\alpha\tau &= \frac{M^\star - \cos\varphi^\star}{\omega^\star_c \sin\varphi^\star}
\end{align*}
E considerando che $ |G(j \omega_c^{\star})|_{\mathrm{dB}} + 20 \log M^{\star} = 0$ ed $M_f^{\star} = 180^\circ + \arg(G_e(j \omega_c^{\star})) + \varphi^{\star}$
Procediamo per tentativi sull' $\omega_c^{\star}$ in modo tale da rendere vere queste identità ed ottenere i parametri $M^{\star}$ e $\varphi^{\star}$.
Dopo vari tentativi, i parametri ottimali sono risultati essere:
\begin{itemize}
    \item $\omega_c^{\star} = 1.1 \cdot \omega_\text{c,min} \approx 2.1831 \cdot 10^3 $, la pulsazione in cui l'ampiezza in dB del sistema esteso si annulla
    \item $\text{Tolleranza di fase} = 8^\circ $ che andrà aggiunta a $M_f^{\star}$ calcolato nel punto 3.3.1 
\end{itemize}
Ottenendo dei valori:
\begin{itemize}
    \item $M^{\star} \approx 2.3708 \cdot 10^4$
    \item $\varphi^{\star} \approx 58.3868 ^\circ $
\end{itemize}
Procediamo quindi al calcolo di $\tau$ ed $\alpha$ della rete anticipatrice:
\begin{itemize}
    \item $\tau \approx 12.7498$
    \item $\alpha \approx 2.2112 \cdot 10^{-5}$
\end{itemize}
Ricavando una rete anticipatrice nella forma:
\begin{equation}
R_{d,\text{anticipatrice}}(s) = \frac{12.75 \cdot s + 1}{2.819 \cdot 10^{-4} s + 1}
\end{equation}
\subsubsection{Sintesi per abbattimento errori di misura}
Per ridurre gli errori di misura, è necessario attenuare i segnali indesiderati ad alta frequenza. A tal fine, aggiungiamo un polo posizionato ad alte frequenze, secondo la forma:
\begin{equation}
R_\text{d,polo} = \frac{1}{(1 + T_1 s)}
\end{equation}
Con $T_1$ il polo calcolato procedendo per tentativi. Otteniamo quindi:
\begin{itemize}
\item \( T_1 \approx 3 \cdot 10^{-5} \)
\end{itemize}
L'inserimento di questo polo consente di portare il diagramma di Bode sotto la zona proibita, attenuando efficacemente l'influenza degli errori di misura.
\begin{equation}
R_\text{d,polo} = \frac{1}{(1 + 3 \cdot 10^{-5} s)}
\end{equation}
\subsubsection{Sintesi regolatore dinamico completa}
Il calcolo ci permette di ottenere un regolatore dinamico del tipo:
\begin{equation}
R_d(s) = \frac{1.275 \cdot 10^1 s + 1}{8.458 \cdot 10^{-9} s^2 + 3.119 \cdot 10^{-4} s + 1}
\end{equation}

% SINESI COMPLETA
\subsection{Sintesi regolatore completa}
Il regolatore finale risulta essere:
\begin{equation}
R(s) = \underbrace{R_s}_{\text{Per errori misura e disturbi}} \cdot 
\underbrace{R_d(s)}_{\text{Polo ad alta frequenza e rete anticipatrice} }
\end{equation}
Dal calcolo, otteniamo:
\begin{equation}
R(s) = \frac{1.567 \cdot 10^6 \, s + 1.229 \cdot 10^5}{8.458 \cdot 10^{-9} \, s^2 + 3.119 \cdot 10^{-4} \, s + 1}
\end{equation}
\begin{figure}[H]
\centering
\includegraphics[width=1\textwidth]{figs/Bode_R(s).jpeg}
\caption{Diagramma di Bode della funzione in anello aperto $R(j\omega)$}
\label{fig:etichetta}
\end{figure}

% FUNZIONE AD ANELLO
\subsection{Funzione ad anello aperto}
La funzione in anello aperto $L(s)$ soddisfa ora le specifiche del sistema:
\begin{figure}[H]
\centering
\includegraphics[width=1\textwidth]{figs/Bode_L(s).jpeg}
\caption{Diagramma di Bode della funzione in anello aperto $L(j\omega)$}
\label{fig:etichetta}
\end{figure}

\section{Test sul sistema}
% LINEARE
\subsection{Test sul sistema linearizzato}
\subsubsection{Test senza disturbi}
Per testare il comportamento del regolatore applicato al sistema linearizzato, iniziamo con il considerare un gradino con equazione $w(t) = 1.5(t)$ come segnale di ingresso. Otteniamo un risultato di questo tipo:
\begin{figure}[H]
    \centering
    \includegraphics[width=0.7\textwidth]{figs/lineare_non_disturbi.jpeg}
    \caption{Simulazione di un gradino $w(t) = 1.5(t)$ in ingresso al sistema linearizzato}
    \label{fig:etichetta}
\end{figure}
Il grafico della risposta evidenzia che il regolatore progettato è efficace, soddisfacendo sia la specifica sulla sovraelongazione che quella relativa al tempo di assestamento.
Le prestazioni del sistema risultano ottimali, rispettando il vincolo temporale di assestamento in meno di 0.003 secondi.

\subsubsection{Test con disturbi}
Applichiamo ora i disturbi come da specifica della traccia:
\begin{figure}[H]
    \centering
    \begin{subfigure}[H]{0.45\textwidth}
        \centering
        \includegraphics[width=\textwidth]{figs/risposta_d(t).jpeg}
        \caption{Sistema lineare con $d(t) = \sum_{k=1}^{4} 0.2 \sin(0.1 k t)$.}
        \label{fig:subfig1}
    \end{subfigure}
    \hfill
    \begin{subfigure}[H]{0.45\textwidth}
        \centering
        \includegraphics[width=\textwidth]{figs/risposta_n(t).jpeg}
        \caption{Sistema lineare con $n(t) = \sum_{k=1}^{4} 0.2 \sin(2 \cdot 10^5 k t)$.}
        \label{fig:subfig2}
    \end{subfigure}
    \label{fig:etichetta_generale}
\end{figure}

\begin{figure}[H]
    \centering
    \includegraphics[width=1\textwidth]{figs/risposta_totale.jpeg}
    \caption{Risposta totale del sistema con entrambi i disturbi applicati}
    \label{fig:etichetta}
\end{figure}

Il grafico della risposta evidenzia che il regolatore progettato è efficace anche dopo l'applicazione dei disturbi, riuscendo a mantenere una sovraelongazione controllata e a rispettare il margine di errore. 

\subsection{Test sul sistema non lineare}
Una volta verificata l'efficacia del regolatore sul sistema linearizzato, vengono effettuate delle prove utilizzando gli stessi ingressi e disturbi per osservare la risposta del sistema non lineare.
Il sistema in retroazione è stato modellato utilizzando Simulink:

\begin{figure}[H]
    \centering
    \includegraphics[width=1\textwidth]{figs/rispostaNonLineare.png}
    \caption{Simulazione grafica del sistema linearizzato con disturbi applicati}
    \label{fig:etichetta}
\end{figure}

\section{Punti Opzionali}
\subsection{Realizzazione interfaccia grafica}
Con l'utilizzo di Processing e p5.js, è stata sviluppata una simulazione grafica interattiva del sistema, arricchita da uno slider che consente di selezionare la posizione angolare $\theta$ della tavola rotante.
Abbiamo utilizzato un delta temporale discreto $\Delta t = 1 \cdot 10^{-7}$ e il \href{https://it.wikipedia.org/wiki/Metodo_di_Eulero}{Metodo di Eulero} per approssimare l'evoluzione del sistema
\begin{figure}[H]
    \centering
    \includegraphics[width=1\textwidth]{figs/grafica.jpeg}
    \caption{Simulazione grafica del sistema}
    \label{fig:etichetta}
\end{figure}

La simulazione è accessibile al seguente link: \href{https://marziano.top/cat.html}{Pagina Web}.
\subsection{Analisi range di condizioni dell'ampiezza del riferimento a gradino sul non lineare}
Applicando in ingresso al sistema non lineare gradini di ampiezze progressivamente crescenti di equazione $w(t) = A_g1(t)$, osserviamo un andamento di questo tipo:

\begin{figure}[H]
    \centering
    \begin{subfigure}{0.45\textwidth}
        \includegraphics[width=\textwidth]{figs/rispostaNonLineare_GradinoAmpiezza5.png}
        \caption{Risposta un gradino di ampiezza $A_g$ = 5}
    \end{subfigure}
    \hfill
    \begin{subfigure}{0.45\textwidth}
        \includegraphics[width=\textwidth]{figs/rispostaNonLineare_GradinoAmpiezza50.png}
        \caption{Risposta un gradino di ampiezza $A_g$ = 50}
    \end{subfigure}

    \bigskip

    \begin{subfigure}{0.45\textwidth}
        \includegraphics[width=\textwidth]{figs/rispostaNonLineare_GradinoAmpiezza800.png}
        \caption{Risposta un gradino di ampiezza $A_g$ = 800}
    \end{subfigure}
    \hfill
    \begin{subfigure}{0.45\textwidth}
        \includegraphics[width=\textwidth]{figs/rispostaNonLineare_GradinoAmpiezza100.png}
        \caption{Risposta un gradino di ampiezza $A_g$ = 100}
    \end{subfigure}

    \centering
    \begin{subfigure}{0.5\textwidth}
        \includegraphics[width=\textwidth]{figs/rispostaNonLineare_GradinoAmpiezza1000.png}
        \caption{Risposta un gradino di ampiezza $A_g$ = 1000}
    \end{subfigure}
    
    \caption{Simulazioni con gradini progressivamente maggiori}
    \label{fig:griglia-immagini}
\end{figure}

La sovraelongazione del sistema aumenta proporzionalmente all'ampiezza del gradino (nonostante resti ugualmente nel range percentuale desiderato).
Tuttavia, dopo aver dato in ingresso un gradino di ampiezza molto alta ($\approx 1000$) l'output finale del sistema risulta chiaramente instabile.

\section{Conclusioni}
Il lavoro svolto è stato strutturato principalmente da una fase inziale di analisi del sistema, una fase di realizzazione del regolatore e infine una fase di testing.
Il sistema meccanico, essendo non lineare, è stato linearizzato attorno al punto di equilibrio. Questo ha permesso di studiarne il comportamento mediante strumenti propri dei sistemi LTI come il calcolo della \textbf{funzione di trasferimento}, permettendoci di astrarre il problema dal modello puramente "fisico".
\\
\\
Il regolatore progettato, composto da una parte \textbf{statica} per garantire precisione sull'errore a regime, e da una parte \textbf{dinamica} per rispettare le restanti specifiche, ha mostrato buone prestazioni sul sistema linearizzato, soddisfacendo i vincoli progettuali.
Test con ingressi a gradino e disturbi sinusoidali ne hanno confermato la robustezza e la precisione.
\\
\\
Tuttavia, applicando il regolatore al sistema \textbf{non linearizzato} ricostruito su Simulink, sono emerse limitazioni significative. In assenza di disturbi, si osserva un aumento del tempo di assestamento e della sovraelongazione, particolarmente marcato aumentando l'ampiezza del gradino inserito.
Con l'introduzione di gradini di ampiezza troppo ampia il sistema può addirittura divergere, mostrando quindi un'incapacità del regolatore di garantire la stabilità del sistema non lineare.
\\
\\
Queste differenze derivano dalla natura fortemente \textbf{non lineare del sistema}, che la linearizzazione non riesce a rappresentare con sufficiente accuratezza.
La progettazione di un regolatore più sofisticato potrebbe migliorare le prestazioni e garantire una maggiore robustezza rispetto alle complessità del sistema reale.
\end{document}
